\documentclass[a4paper,10pt]{article}
%\documentclass[a4paper,10pt]{scrartcl}

\usepackage[utf8]{inputenc}

\title{}
\author{}
\date{}

\pdfinfo{%
  /Title    ()
  /Author   ()
  /Creator  ()
  /Producer ()
  /Subject  ()
  /Keywords ()
}

\begin{document}
\maketitle

\paragraph{Arithmetic constraints :}
$$
\begin{array}{rcl}
   \phi &::=   &  x{=}\kappa^{(i)}(x_1,\ldots,x_k)  \quad\mid\quad x=x_1+x_2
                 \quad\mid\quad x=x_1\cdot x_2 \quad\mid       \\
           &&  x_1=x_2  \quad\mid\quad x\in S \quad\mid\quad \phi \wedge \phi' \quad\mid\quad \exists x.\phi
\end{array}
$$ 

where $i\in \Nat$, $\kappa:\PosReal^k\to\PosReal$, $x,x_1,x_2$ are variables, and $S\subseteq\PosReal$ a finite set.
A solution of an arithmetic constraint $\phi$ with $n$ variables can be identified with a tuple in $\PosReal^n$ since we assumed a total
order on the variables. Therefore, the solution set $\Sol(\phi)$ of a formula $\phi$ satisfies $\Sol(\phi)\subseteq \PosReal^n$.


\begin{lemma}
The steady state equations of any reaction network
$N$ can be rewritten in linear time to an equivalent
arithmetic constraint $\phi_N$ so that:
       $$\Sol(\phi_N) = \ExR_N$$
\end{lemma}

\begin{proof}
It is sufficient to flatten the terms in steady state equations, 
by introducing new existentially bound variables for all nested 
subterms. The resulting constraint $\phi_N$  is clearly equivalent, 
and thus it satisfies $\Sol(\phi_N) = \ExR_N$ by Definition
\ref{def:ExR} of the exchange relation.
\end{proof}

\paragraph{A difference constraint :}
where $x$ is a variable, $d \in \PosReals$, 
$S\subseteq \PosReals$, and  $\kappa:\PosReals^k\to \PosReals$:
$$
\begin{array}{l@{\quad}rcl}
\text{difference relation} &   t  &  ::=& x \mid d \\
\text{set of difference relations} 
            &   s   & ::=& t \mid S \mid s+s' \mid s\cdot s'
                                   \mid \kappa(s_1,\ldots,s_k)  \\
\text{difference constraints} &   \psi &::=& t \in s \mid 
            t {=}t' \mid
            \psi \wedge \psi' \mid \exists x.\psi
\end{array}
$$


\begin{figure}
\joachiminline{
\begin{tabular}{l}
Difference relations: \\[.5em]
\qquad $\SemEle{x} = \alpha(x)   $  \\
\qquad $\SemEle{d} = \mbox{ unique element of  } \DIFF{\{d\}}$  
 \\[.5em]
Sets of difference relations:\\[.5em]
\qquad $\begin{array}{ll}
\SemSet{t}  = \{ \SemEle{t} \} & 
 \SemSet{S} = \DIFF{S}  
\\
\SemSet{s\cdot s'} = \SemSet{s} \DIFF{\cdot}  \SemSet{s'} &
  \SemSet{s+ s'} = \SemSet{s} \DIFF{+}  \SemSet{s'}
\\
 \multicolumn{2}{l}{\SemSet{\kappa(s_1,\ldots,s_k)} = \DIFF{\kappa}
  (\SemSet{s_1},\ldots,\SemSet{s_k})}
\\[.5em]
\end{array}
$
\\
Difference constraints:\\[.5em]
\qquad $\begin{array}{ll}
  \SemBool{t\in s} = (\SemEle{t} \in \SemSet{s}) &
  \SemBool{t=t'} = (\SemEle{t} = \SemEle{t'}) 
\\
  \SemBool{\psi \wedge\psi'} = \SemBool{\psi} \wedge^\Bool  \SemBool{\psi'} 
&
  \SemBool{\exists x.\psi} = (\exists
  \delta\in\Delta. \SemBoolAlpha{\psi}{\alpha[x/\delta]} = 1) 
\end{array}
$
\end{tabular}
} %\joachiminline
\caption{\label{sem:dc}Semantics of difference constraints over
  $\Delta$.}
\end{figure}

The set of all \emph{$\Delta$-solutions} of a difference constraints $\psi$
with free variables $x_1,\ldots, x_n$ -- listed in their order -- is the
following subset of $\Delta^n$:
$$
  \Sol^\Delta(\psi) = \{(\alpha(x_1),\ldots,\alpha(x_n)) \mid
  \SemBool{\psi}= 1\}
$$

\paragraph{Abstract interpretation :}

We can now abstract from arithmetic constraints by
interpreting them as difference constraints:
$$
\begin{array}{l@{\qquad}l}
\Enc{x{=}\kappa^{(i)}(x_1,\ldots,x_k)} =_\DEF x\in
  \kappa(x_1,\ldots,x_k) &
\Enc{x\in S} =_\DEF x \in S 
\\
\Enc{x=x_1+x_2} =_\DEF x\in x_1+ x_2  &
\Enc{\phi\wedge\phi'} =_\DEF \Enc\phi\wedge\Enc{\phi'}   
\\
\Enc{x=x_1\cdot x_2} =_\DEF x\in x_1\cdot x_2 
&
\Enc{\exists x. \phi} =_\DEF \exists x.\Enc{\phi} 
\\
\Enc{x_1{=}x_2} =_\DEF (x_1{=}x_2)   
\end{array}
$$

\begin{theorem}[Soundness]
For any arithmetic constraint $\phi$ and any partition $\Delta$ of
difference relations, the $\Delta$-solution set of the abstract
interpretation of $\psi$ over-approximates the $\Delta$-difference 
abstraction of the solution set of $\phi$, that is:
\label{theo}
 $$\DIFF{\Sol(\phi)} \subseteq \Sol^\Delta(\Enc{\phi}).$$
\end{theorem}

\begin{proof} We first note that for any two relations
$R_1,R_2\subseteq
  \PosReal^n$:
$$\text{(intersect)}\qquad \DIFF{(R_1\cap R_2)}\subseteq \DIFF{R_1}\cap
\DIFF{R_2}$$
This can be verified straightforwardly from the definition
of the difference abstraction. It will turn out to be the 
reason why proper approximations may appear when 
abstracting conjunctions.

Then we proceed by induction on the structure of arithmetic
constraints $\phi$. For the variables in the proof, we always 
assume for simplicity that they are ordered $x_1 x_2\ldots x_k x$.


\begin{description}
\item [Case $\phi$ is $x=\kappa^{(i)}(x_1,\ldots,x_k)$.]
\joachiminline{By definition of solutions we have that $\Sol(\phi)= \{(d_1,$
$\ldots,d_k,d) \in\PosReal^{k+1}
\ \mid\ $$d=\kappa'(d_1,\ldots,d_k),\ \kappa'\sim_\Delta \kappa\}$,
i.e., $\Sol(\phi)=\{ \kappa'\mid
\kappa'\sim_\Delta\kappa\}$. Thus, $ \DIFF{\Sol(\phi)} =
\DIFF{\kappa}$. 
Furthermore, $\Enc{\phi}$ is equal to $x{\in} \kappa(x_1,\ldots,x_k)$
so that $\Sol^\Delta(\Enc{\phi})= \kappa^\Delta=
 \DIFF{\Sol(\phi)}$. }%joachiminline
\item [Case $\phi$ is $x=x_1+x_2$.] 
\joachiminline{This follows, since 
$\DIFF{\SOL\phi} = \DIFF{+}$. Furthermore 
$\Enc{\phi}$ is equal to $x\in x_1+x_2$ so 
that $\Sol^\Delta(\Enc{\phi}) = \DIFF{+}$ and thus $\DIFF{\SOL\phi}
=\Sol^\Delta{(\Enc\phi)}$.} % joachiminline
\item [Case $\phi$ is $x=x_1\cdot x_2$.] 
This follows in analogy to the case of addition.
\item [Case $\phi$ is $x_1=x_2$.] We first note that
$\DIFF{\{(d,d)\mid d\in
  \PosReal\}}=\{(\delta,\delta)\mid\delta\in\Delta\}$
since $\Delta$ is a partition of $\PosReals\times\PosReals$.
We now can conclude as follows:
$\DIFF{\SOL\phi}= \DIFF{\{(d,d)\mid d\in
  \PosReal\}}=\{(\delta,\delta)\mid\delta\in\Delta\} =
\Sol^\Delta{\Enc{\phi}}$.
\item [Case $\phi$ is $x\in S$.] \joachiminline{In this case we have $\Sol(\phi)=S$,
so that $\DIFF{\Sol(\phi)}=\DIFF{S}=\Sol^\Delta(x\in S)=\Sol^\Delta(\Enc{\phi})$.} 
\item [Case $\phi$ is $\phi_1\wedge \phi_2$.] Hence
$\Sol(\phi)=\Sol(\phi_1)\cap\Sol(\phi_2)$ and thus
as claimed by (intersect) at the beginning
$\DIFF{\Sol(\phi)}\subseteq \DIFF{\Sol(\phi_1)}\cap
  \DIFF{\Sol(\phi_2)}$.
From the induction hypothesis applied to $\phi_1$ and $\phi_2$, 
it follows that $\DIFF{\Sol(\phi_i)}\subseteq \Sol(\Enc{\phi_i})$
for $i=1,2$, and thus
$\DIFF{\Sol(\phi)}\subseteq \DIFF{\Sol(\phi_1)}\cap
  \DIFF{\Sol(\phi_2)}\subseteq \Sol^\Delta(\Enc{\phi_1})
\cap \Sol^\Delta(\Enc{\phi_2}) = \Sol^\Delta(\Enc{\phi})$.
\item [Case $\phi$ is $\exists x.\phi'$.]
There case where $x$ does not occur freely in $\phi'$ is
easy, since we can drop the quantifier $\exists x$. Otherwise,
we can assume that $\phi'$ has the free variables
$x_1,...,x_k$ and that $x=x_i$ where $1\le i\le k$.
For any set $D$ and tuple set $S\subseteq D^k$ we
define the projection $\Pi_x(S)=\{(d_1,\ldots, d_{i-1},d_{i+1},\ldots,d_k)\mid
(d_1,\ldots,d_k)\in S\}$. We first note that: $$\text{(project)}\qquad \Pi_x
\DIFF{(\Sol(\phi'))} 
= \DIFF{(\Pi_x(\Sol(\phi')))}$$
All elements of $\PosReal\times\PosReal$ belong to some
difference in $\Delta$.
Based on the equation on projection, we can conclude as follows:
$$
\begin{array}{ll}
\DIFF{\Sol(\exists x.\phi')} &= \DIFF{(\Pi_x(\Sol(\phi')))}
=_{project} \Pi_x \DIFF{(\Sol(\phi'))} 
\\
&\subseteq_{ind. hyp.} 
\Pi_x \Sol(\Enc{\phi'}) = \Sol(\exists x.\Enc{\phi'}) 
= \Sol^\Delta(\Enc{\exists x.\phi'}).
\end{array}$$
\end{description}
\end{proof}

\begin{corollary}\label{corollary:inclusion} If
  $\ExR_N=\Sol(\phi_N)$ then $\DIFF{(\ExR_N)}\subseteq\Sol^\Delta(\Enc{\phi_N})$.
\end{corollary}
%
\begin{proof} 
This follows immediately from Theorem \ref{theo}.
%, since $\ExR_N= \Sol(\phi_N)$ by construction of $\phi_N$.
\end{proof}

\begin{definition}
The set of \emph{$\Delta$-difference formulas} $\Psi$ is the least set
that contains all difference constraints $\phi$, all formulas $x\in \Delta'$ 
where $x$ is a variable and $\Delta'\subseteq \Delta$, and
that is closed under the first-order connectives, ie.:
$$
   \Psi ::= \phi \mid x\in \Delta' \mid \Psi\wedge \Psi' \mid \neg
   \Psi \mid \exists x.\Psi
$$
\end{definition}
We define the $\Delta$-difference formula $x=\delta$ as syntactic 
sugar for $x\in\{\delta\}$  for any $\delta\in\Delta$ and variable $x$.

\end{document}
